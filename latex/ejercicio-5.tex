\textit{Diseñar e implementar un algoritmo para CIDM que use la metaheurística GRASP.}

\subsection{Ejercicio A}

\textit{Explicar detalladamente el algoritmo implementado. Plantear distintos criterios de parada y de selección de la lista de candidatos (RCL) de la heurística golosa aleatorizada.}

\medskip

\subsubsection{Idea general}

Como pide el enunciado, la idea general del algoritmo es usar la metaheurística GRASP, para lo cual es necesario tener implementaciones de: heurística constructiva golosa y heurística de búsqueda local, que fueron conveniemente implementadas en los puntos anteriores.
Mostramos la estructura general de un algoritmo GRASP:

\begin{codesnippet}
Poner en mejor_solucion una primera solucion Random.
Mientras no se cumpla el criterio de parada hacer:
    Poner en nueva_solucion una solucion usando la funcion ConstruirGreedyRandom()
    Intentar mejorar la nueva_solucion usando la funcion BusquedaLocal()
    Si costo(nueva_solucion) < costo(mejor_solucion) hacer:
        Poner en mejor_solucion la nueva_solucion
\end{codesnippet}

\subsubsection{Criterios de parada}

\subsubsection{Selección de lista de candidatos (RCL)}

\subsubsection{Pseudocódigo}
\begin{codesnippet}
funcion resolver:
    Creamos un vector de n elementos, llenandolo con los nodos desde 0 a n-1.
    llamamos a resolver_aux pasandole como parametro la matriz de adyacencia, el vector
        recién creado y la cantidad de nodos en el grafo.
\end{codesnippet}

\subsection{Ejercicio B}

\textit{Realizar una experimentación que permita observar los tiempos de ejecución y la calidad de las soluciones obtenidas.}
\medskip
