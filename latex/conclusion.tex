Una vez analizados los resultados en las distintas familias de grafos que fueron experimentadas, podemos llegar a las siguientes conclusiones sobre las mismas:

\begin{itemize}

\item El algoritmo goloso es el más rápido en tiempos de ejecución, sin embargo, la solución arrojada no siempre es la óptima. Más aún, como vimos en el ejercicio 3, la solución del goloso aunque rápida puede ser muy mala.

\item El algoritmo GRASP es una buena solución media, aunque sus tiempos de ejecución son más lentos que el goloso, sus soluciones suelen ser mejores.

\item Para las familias evaluadas, el mejor algoritmo fue el de búsqueda local ya que arrojo no solo los mejores resultados sino unos tiempos de ejecución muy aceptables y sólo superados por los de la solución golosa, que presenta peores resultados.

\item Para todo grafo con $n \leq 13$ lo más conveniente sería usar el algoritmo exacto, ya que aunque es exponencial y mucho más lento que los otros, cuando la entrada es manejablemente pequeña uno se puede garantizar tener el resultado óptimo en un tiempo aceptable.

\end{itemize}

En conclusión, podemos ver que los algoritmos heurísticos son de gran ayuda para encarar problemas NP ya que, aunque no siempre arrojan la solución óptima, trabajan en tiempo polinomial, acortando así enormemente los tiempos de ejecución del programa. Más aún, pudimos ver que las soluciones arrojadas por algunos de los algoritmos como busqueda local y GRASP además de correr en tiempo polinomial arrojan soluciones que pueden ser óptimas o cercanas a las óptimas. Y, aunque esto no significa tener el resultado exacto, nos permite encarar un problema exponencial y conseguir una solución rapidamente.